% ---
% Conclusão (outro exemplo de capítulo sem numeração e presente no sumário)
% Dependendo do trabalho desenvolvido ele pode ter uma Conclusão ou Considerações finais
% Para trabalhos de disciplina utilizar Considerações Finais
% ---
\chapter*{Considerações Finais}
\addcontentsline{toc}{chapter}{Considerações Finais}
% Para definir com número utilizar sem o asterisco
%\chapter{Considerações finais}


% ---
Além desse documento ser um modelo de como pode ser criado um documento em \LaTeX \space ele também apresenta diversas informações úteis para as disciplinas de projetos de informática do \ac{ifsp} e alguns elementos uteis para as monografias do curso de Pós Graduação em Gestão de TI do \ac{ifsp}.



\explicacao{Exemplo de seções para monografia da pós graduação...}
\section{Resposta à Questão de Pesquisa}
\lipsum[3-5]

\section{Objetivos Propostos}
\lipsum[3-5]

\section{Contribuições Acadêmicas e Gerenciais}
\lipsum[3-5]

\section{Limitações da Pesquisa e Contribuições para Estudo}
\lipsum[3-5]
