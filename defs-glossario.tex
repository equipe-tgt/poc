
% Definições para glossario

% ATENCAO o SHARELATEX GERA O GLOSSARIO/LISTAS DE SIGLAS SOMENTE UMA VEZ
% CASO SEJA FEITA ALGUMA ALTERAÇÃO NA LISTA DE SIGLAS OU GLOSSARIO É NECESSARIO UTILIZAR A OPÇÃO :
% "Clear Cached Files" DISPONIVEL NA VISUALIZAÇÃO DOS LOGS 
% ---
% https://www.sharelatex.com/learn/Glossaries


\newglossaryentry{backend}{
                name={Back-end},
                description={Um sistema \emph{back-end} é aquele que se encontra na
    camada de servidor, em uma aplicação de arquitetura cliente-servidor. Sua
    principal função é fornecer informações e capacidade de
    processamento a aplicação cliente.}}
    
\newglossaryentry{build}{
                name={Build},
                description={Termo usado para identificar uma versão compilada de um programa}}    
    

\newglossaryentry{branch} {
    name=branch,
    plural= {branches},
    description={Uma ramificação em controles de versão são ponteiros para as alterações feitas nos arquivos do projeto}
}        

\newglossaryentry{commit} {
    name=commit,
    plural= {commits},
    description={Ato de enviar e guardar, ou seja, enviar dados ou códigos para armazenamento em um banco de dados ou em um sistema de controle de versão}
}    
    
\newglossaryentry{deploy} {
    name=Deploy,
    plural= {deploys},
    description={Processo pelo qual a aplicação é implantada em ambiente
  de prpdução, e está disponível para os usuários finais.}
}

\newglossaryentry{framework} {
    name=Framework,
    description={Conjunto de ações e estratégias que possuem um ojeto e objetivo em específico;
    Na programação, é um conjunto de pacotes e bibliotecas que abstarai
    alguma função complexa, geralmente de nível mais baixo, e sobre o
    qual uma aplicação pdoe ser contruída.}
}

\newglossaryentry{frontend} {
    name={Front-end},
    description={Um sistema \emph{front-end} é aquele que encontra na
    camada cliente, em uma aplicação de arquitetura cliente-servidor. Sua
    principal função, no escopo deste projeto, é atuar como interface
    gráfica para o usário, coletar dados e enviá-los para o \emph{back-end}.}
}    

\newglossaryentry{pipeline} {
    name={pipeline},
    plural={pipelines},
    description={Série de passos automatizados que devem ser implementados para entregar uma nova versão de um software}
}


\newglossaryentry{rollback} {
    name={rollback},
    description={Processo de retornar um banco de dados ou um programa para um estado anterior, geralmente utilizado em recuperação de um estado antes de erros acontecerem}
}
                
% Normalmente somente as palavras referenciadas são impressas no glossario, portanto é necessário referenciar utilizando \gls{identificação}                
