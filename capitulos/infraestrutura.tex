\chapter{Infraestrutura}

Visando implementar a aplicação conforme descrita no Desenho de
Porjeto, foi desenvolvida uma infraestrutura usando os serviços
disponibilizados pela \gls{aws}.  Mas, num desvio em relação
aquele documento, a equipe optou por usar a ferramenta Github Actions
para implementar a \gls{pipeline} de \gls{deploy}.  Usando esta \gls{pipeline}, uma
imagem de Docker é gerada com o executável do aplicativo, isolado em
um ambiente de execução e com a porta de conexão exposta.

Esta imagem é, então, enviada ao \gls{aws} Elastic Container Registry, que
mantém um histótico destas imagens, e a partir deste registry, uma
tarefa é iniciada e colocada num container, contruídos de modo a
complementar a imagem (expondo as portas esperadas pela imagem e pela
aplicação, por exemplo). Este container é gerado e gerenciado de forma
automática, tendo politicas de segurança associados a ele.

A partir deste ponto, o container se comunica com duas outras
ferramentas: um cluster serverless de banco de dados aurora,
compatível com MySQL 5, no \gls{aws} Relational Database Service, serviço
que gerencia as instâncias de banco de dados de forma automática, e
permite acesso a elas a partir de grupos de segurança
específicos. Nestes grupos de segurança estão os containers da
aplicação \gls{backend}, o que permite a comunicação entre ambas as
aplicações.

No outro lado da aplicação \gls{backend}, o \gls{aws} Elastic Load Balancer
oferece um ponto de comunicação entre o a rede privada, onde o \gls{backend}
e o banco de dados se encontram, e a internet externa. Este serviço
redireciona as requisições \gls{https} externas entre todas as instâncias da
aplicação que estejam rodando.

Alguns pontos a ser melhor desenvolvidos na infraestrutura são:
\begin{itemize}
\item Manter a instâncias da aplicação \gls{backend} ativas e estáveis por
  um período maior, de modo que pelo menos uma possa responder a
  qualquer momento;
\item Usar o protocolo \gls{https}, como está estabelecido nos requisitos de
  segurança;
\item Implementar outras \glspl{pipeline}, para escalar a aplicação e
  realizar o \gls{rollback} a uma versão anterior;
\item Isolar os ambientes de teste e de produção;
\item Usar o serviço de gerenciamento de segredos para armazenar
  senhas de banco de dados;
\item Proteger a \gls{branch} master do repositório do Github, para
  que esta aceite novos \glspl{commit} apenas a partir de Pull Requests, uma
  vez que um \gls{commit} novo nesta é o trigger para a execução da \gls{pipeline}
  no Github Actions.
\end{itemize}