\chapter{Tecnologias}

Nesse capítulo, serão abordados as principais tecnologias e seus casos de uso no sistema.

\section{Front-end}
Para o desenvolvimento do \gls{frontend} da aplicação estamos utilizando a linguagem Javascript, a \gls{framework} React-Native e estilizando a interface com a biblioteca Native-Base.
Além destas tecnologias, utilizamos as seguintes ferramentas da \gls{framework} Expo:

\begin{itemize}
	\item \textit{Expo CLI}:
		\begin{itemize}
			\item Possibilita a criação de novos projetos de React-Native com estrutura básica inclusa;
			\item Possibilita a execução da aplicação em modo de desenvolvimento com recarregamento rápido, viabilizando a emulação no navegador, em dispositivos móveis (a partir do Expo Go) e em emuladores de dispositivos móveis;
			
			\item Possibilita fazer o \gls{build} dos arquivos binários \gls{apk} e \gls{ipa};
		\end{itemize}
		
			\item \textit{Expo Go}:
		\begin{itemize}
			\item Possibilita a execução da aplicação em desenvolvimento em dispositivos móveis antes do \gls{deploy}
		\end{itemize}
\end{itemize}

\section{Back-end}
Para o desenvolvimento do backend da aplicação, foi utilizado a linguagem Java com o framework Spring e todo seu ecossistema de tecnologia disponível.

As principais tecnologias do backend são:

\begin{itemize}

	\item \underline{Lombok}: Automatiza métodos amplamente utilizados no java, tais como \textit{getters}, \textit{setters} e construtores.

	\item \underline{Hibernate}: Permite a conexão com o banco de dados através de uma abstração de alto nível.	
	
	\item \underline{Hibernate-Spatial}: É um complemento do hibernate, permitindo a integração com os dados do tipo espacial, permitindo geolocalização.
	
	\item \underline{Spring-Data}: É uma ferramenta do ecossistema Spring que permite uma forma mais prática de comunicar a api com o banco de dados através do hibernate, fornecendo uma abstração mais prática do que a do hibernate clássico.
	
	\item \underline{Flyway}: É um versionador de scripts de banco de dados.
	
	\item \underline{OAuth2}: É uma tecnologia que fornece um servidor de autenticação e autorização, garantindo alta segurança e alta escalabilidade.
	
	\item \underline{Spring Security}: É uma ferramenta do ecossistema Spring que fornece recursos necessários para proteger uma api.
	
	\item \underline{Javamail}: É um plugin que permite o envio de email dentro da nossa própria api.
	
	\item \underline{Swagger}: É um plugin que permite a documentação da api.
	
\end{itemize}

