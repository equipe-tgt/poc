%% Adaptado a partir de :
%%    abtex2-modelo-trabalho-academico.tex, v-1.9.2 laurocesar
%% para ser um modelo para os trabalhos no IFSP-SPO

\documentclass[
    % -- opções da classe memoir --
    12pt,               % tamanho da fonte
    openright,          % capítulos começam em pág ímpar (insere página vazia caso preciso)
    %twoside,            % para impressão em verso e anverso. Oposto a oneside
    oneside,
    a4paper,            % tamanho do papel. 
    % -- opções da classe abntex2 --schwinn
    % Opções que não devem ser utilizadas na versão final do documento
    %draft,              % para compilar mais rápido, remover na versão final
    paginasA3,  % indica que vai utilizar paginas em A3 
    MODELO,             % indica que é um documento modelo então precisa dos geradores de texto
    TODO,               % indica que deve apresentar lista de pendencias 
    % -- opções do pacote babel --
    english,            % idioma adicional para hifenização
    brazil              % o último idioma é o principal do documento
    ]{ifsp-spo-inf-ctds} % ajustar de acordo com o modelo desejado para o curso



% ---
% Informações de dados para CAPA e FOLHA DE ROSTO
% ---
\titulo{Prova de Conceito Lixt}

% Trabalho individual
%\autor{AUTOR DO TRABALHO}

% Trabalho em Equipe
% ver também https://github.com/abntex/abntex2/wiki/FAQ#como-adicionar-mais-de-um-autor-ao-meu-projeto
\renewcommand{\imprimirautor}{
\begin{tabular}{lr}
Alkindar José Ferraz Rodrigues & SP3029956 \\
Carolina de Moraes Josephik & SP3030571 \\
Fabio Mendes Torres & SP3023184 \\
Gabriely de Jesus Santos Bicigo & SP303061X \\
Leonardo Naoki Narita & SP3022498 \\
Mariana da Silva Zangrossi & SP3030679 \\
\end{tabular}
}

\disciplina{PI1 - Projeto Integrado I}

\preambulo{Prova de conceito da aplicação Lixt da 
disciplina de Projeto Integrado I no 1°
semestre de 2021.}

\data{29 de Junho de 2021}

% Definir o que for necessário e comentar o que não for necessário
% Utilizar o Nome Completo, abntex tem orientador e coorientador
% então vão ser utilizados na definição de professor
\renewcommand{\orientadorname}{Professor:}
\orientador{Ivan Francolin Martinez}
\renewcommand{\coorientadorname}{Professor:}
\coorientador{José Braz de Araujo}

% ---


% informações do PDF
\makeatletter
\hypersetup{
        %pagebackref=true,
        pdftitle={\@title}, 
        pdfauthor={\@author},
        pdfsubject={\imprimirpreambulo},
        pdfcreator={LaTeX with abnTeX2 using IFSP model},
        pdfkeywords={abnt}{latex}{abntex}{abntex2}{IFSP}{\ifspprefixo}{trabalho acadêmico}, 
        colorlinks=true,            % false: boxed links; true: colored links
        linkcolor=blue,             % color of internal links
        citecolor=blue,             % color of links to bibliography
        filecolor=magenta,              % color of file links
        urlcolor=blue,
        bookmarksdepth=4
}
\makeatother
% --- 

% carregando aqui referencias quando utilizando BIBLATEX
\IfPackageLoaded{biblatex}{%
\addbibresource{referencias.bib}
\addbibresource{exemplos/abntex2-doc-abnt-6023.bib}
}{}

% ----
% Início do documento
% ----
\begin{document}


% Retira espaço extra obsoleto entre as frases.
\frenchspacing 

\newpage

% ----------------------------------------------------------
% ELEMENTOS PRÉ-TEXTUAIS
% ----------------------------------------------------------
\pretextual

% ---
% Capa
% ---
\imprimircapa

% ---

% ---
% Folha de rosto
% (o * indica que haverá a ficha bibliográfica)
% ---
\imprimirfolhaderosto
%\imprimirfolhaderosto*
% ---

% -- resumo obrigatório
%% ---
% RESUMOS
% ---

% resumo em português
\setlength{\absparsep}{18pt} % ajusta o espaçamento dos parágrafos do resumo
\begin{resumo}

 Segundo a \citeonline[3.1-3.2]{NBR6028:2003}\index{ABNT}\index{NBR6028}, o resumo\index{resumo} deve ressaltar o contexto, o objetivo, o método, os resultados e as conclusões do documento (portanto deve ser escrito por ultimo). A ordem e a extensão destes itens dependem do tipo de resumo (informativo ou indicativo) e do  tratamento que cada item recebe no documento original. O resumo \textbf{deve ter um paragrafo único} e deve \textbf{ter entre 150 e 500 palavras para trabalhos acadêmicos ou entre 100 e 250 para artigos de periódicos}. O resumo deve ser  precedido da referência do documento, com exceção do resumo inserido no  próprio documento. (\ldots) As palavras-chave devem figurar logo abaixo do resumo, antecedidas da expressão \textbf{Palavras-chave}:, separadas entre si por ponto e finalizadas também por ponto.

 \textbf{Palavras-chaves}: latex. abntex. editoração de texto.
\end{resumo}

% resumo em inglês
\begin{resumo}[Abstract]
 \begin{otherlanguage*}{english}
   This is the english abstract.
   \vspace{\onelineskip}

   \noindent 
   \textbf{Keywords}: latex. abntex. text editoration.
 \end{otherlanguage*}
\end{resumo}


% ---
% inserir lista de ilustrações
% ---
%\pdfbookmark[0]{\listfigurename}{lof}
%\listoffigures*
%\cleardoublepage
% ---

% ---
% inserir lista de tabelas
% ---
%\pdfbookmark[0]{\listtablename}{lot}
%\listoftables*
%\cleardoublepage
% ---

% ---
% inserir lista de quadros
% ---
%\pdfbookmark[0]{\listofquadrosname}{loq}
%\listofquadros*
%\cleardoublepage
% ---

% ---
% inserir lista de abreviaturas e siglas
% ATENCAO o SHARELATEX/OVERLEAF GERA O GLOSSARIO SOMENTE UMA VEZ
% CASO SEJA FEITA ALGUMA ALTERAÇÃO NA LISTA DE SIGLAS É NECESSARIO UTILIZAR A OPÇÃO :
% "Clear Cached Files" DISPONIVEL NA VISUALIZAÇÃO DOS LOGS 
% ---
% https://www.sharelatex.com/learn/Glossaries


\ifdef{\printnoidxglossary}{
    \printnoidxglossary[type=\acronymtype,title=Lista de abreviaturas e siglas,style=siglas]
    \cleardoublepage
}{}


% ---
% inserir o sumario
% ---
\pdfbookmark[0]{\contentsname}{toc}
\tableofcontents
\cleardoublepage
% ---

% ----------------------------------------------------------
% ELEMENTOS TEXTUAIS
% ----------------------------------------------------------
\textual

% Para facilitar a manutenção é sempre melhore criar um arquivo por capitulo, para exemplo isso não é necessário 
\chapter{Arquitetura}

\chapter{Tecnologias}


\chapter{Infraestrutura}




% ----------------------------------------------------------
% Finaliza a parte no bookmark do PDF
% para que se inicie o bookmark na raiz
% e adiciona espaço de parte no Sumário
% ----------------------------------------------------------
\phantompart

% ----------------------------------------------------------
% ELEMENTOS PÓS-TEXTUAIS
% ----------------------------------------------------------
\postextual
% ----------------------------------------------------------

% ----------------------------------------------------------
% Referências bibliográficas
% ----------------------------------------------------------
% quando não esta utilizando biblatex tem que carregar as referencias aqui
%\IfPackageLoaded{biblatex}{}{%
%\bibliography{referencias,exemplos/abntex2-doc-abnt-6023}
%}

% ----------------------------------------------------------
% Glossário
% ----------------------------------------------------------
%
%
\ifdef{\printnoidxglossary}{
    \addcontentsline{toc}{chapter}{GLOSSÁRIO}
    \printnoidxglossary[style=glossario]
    %\printglossaries
}{}

\end{document}