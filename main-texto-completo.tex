%% Adaptado a partir de :
%%    abtex2-modelo-trabalho-academico.tex, v-1.9.2 laurocesar
%% para ser um modelo para os trabalhos no IFSP-SPO

\documentclass[
    % -- opções da classe memoir --
    12pt,               % tamanho da fonte
    openright,          % capítulos começam em pág ímpar (insere página vazia caso preciso)
    %twoside,            % para impressão em verso e anverso. Oposto a oneside
    oneside,
    a4paper,            % tamanho do papel. 
    % -- opções da classe abntex2 --schwinn
    % Opções que não devem ser utilizadas na versão final do documento
    %draft,              % para compilar mais rápido, remover na versão final
    paginasA3,  % indica que vai utilizar paginas em A3 
    MODELO,             % indica que é um documento modelo então precisa dos geradores de texto
    TODO,               % indica que deve apresentar lista de pendencias 
    % -- opções do pacote babel --
    english,            % idioma adicional para hifenização
    brazil              % o último idioma é o principal do documento
    ]{ifsp-spo-inf-ctds} % ajustar de acordo com o modelo desejado para o curso



% ---
% Informações de dados para CAPA e FOLHA DE ROSTO
% ---
\titulo{Prova de Conceito Lixt}

% Trabalho individual
%\autor{AUTOR DO TRABALHO}

% Trabalho em Equipe
% ver também https://github.com/abntex/abntex2/wiki/FAQ#como-adicionar-mais-de-um-autor-ao-meu-projeto
\renewcommand{\imprimirautor}{
\begin{tabular}{lr}
Alkindar José Ferraz Rodrigues & SP3029956 \\
Carolina de Moraes Josephik & SP3030571 \\
Fabio Mendes Torres & SP3023184 \\
Gabriely de Jesus Santos Bicigo & SP303061X \\
Leonardo Naoki Narita & SP3022498 \\
Mariana da Silva Zangrossi & SP3030679 \\
\end{tabular}
}

\disciplina{PI1 - Projeto Integrado I}

\preambulo{Prova de conceito da aplicação Lixt da 
disciplina de Projeto Integrado I no 1°
semestre de 2021.}

\data{29 de Junho de 2021}

% Definir o que for necessário e comentar o que não for necessário
% Utilizar o Nome Completo, abntex tem orientador e coorientador
% então vão ser utilizados na definição de professor
\renewcommand{\orientadorname}{Professor:}
\orientador{Ivan Francolin Martinez}
\renewcommand{\coorientadorname}{Professor:}
\coorientador{José Braz de Araujo}

% ---


% informações do PDF
\makeatletter
\hypersetup{
        %pagebackref=true,
        pdftitle={\@title}, 
        pdfauthor={\@author},
        pdfsubject={\imprimirpreambulo},
        pdfcreator={LaTeX with abnTeX2 using IFSP model},
        pdfkeywords={abnt}{latex}{abntex}{abntex2}{IFSP}{\ifspprefixo}{trabalho acadêmico}, 
        colorlinks=true,            % false: boxed links; true: colored links
        linkcolor=blue,             % color of internal links
        citecolor=blue,             % color of links to bibliography
        filecolor=magenta,              % color of file links
        urlcolor=blue,
        bookmarksdepth=4
}
\makeatother
% --- 

% carregando aqui referencias quando utilizando BIBLATEX
\IfPackageLoaded{biblatex}{%
\addbibresource{referencias.bib}
\addbibresource{exemplos/abntex2-doc-abnt-6023.bib}
}{}

% ----
% Início do documento
% ----
\begin{document}


% Retira espaço extra obsoleto entre as frases.
\frenchspacing 

\newpage

% ----------------------------------------------------------
% ELEMENTOS PRÉ-TEXTUAIS
% ----------------------------------------------------------
\pretextual

% ---
% Capa
% ---
\imprimircapa

% ---

% ---
% Folha de rosto
% (o * indica que haverá a ficha bibliográfica)
% ---
\imprimirfolhaderosto
%\imprimirfolhaderosto*
% ---

% -- resumo obrigatório
%\input{pre-resumos}


% ---
% inserir lista de ilustrações
% ---
\pdfbookmark[0]{\listfigurename}{lof}
\listoffigures*
\cleardoublepage
% ---

% ---
% inserir lista de tabelas
% ---
%\pdfbookmark[0]{\listtablename}{lot}
%\listoftables*
%\cleardoublepage
% ---

% ---
% inserir lista de quadros
% ---
%\pdfbookmark[0]{\listofquadrosname}{loq}
%\listofquadros*
%\cleardoublepage
% ---

\input{pre-siglas}

% ---
% inserir o sumario
% ---
\pdfbookmark[0]{\contentsname}{toc}
\tableofcontents
\cleardoublepage
% ---

% ----------------------------------------------------------
% ELEMENTOS TEXTUAIS
% ----------------------------------------------------------
\textual

% Para facilitar a manutenção é sempre melhore criar um arquivo por capitulo, para exemplo isso não é necessário 
\chapter{Infraestrutura}

Visando implementar a aplicação conforme descrita no Desenho de
Porjeto, foi desenvolvida uma infraestrutura usando os serviços
disponibilizados pela Amazon Web Services.  Mas, num desvio em relação
aquele documento, a equipe optou por usar a ferramenta Github Actions
para implementar a pipeline de deploy.  Usando esta pipeline, uma
imagem de Docker é gerada com o executável do aplicativo, isolado em
um ambiente de execução e com a porta de conexão exposta.

Esta imagem é, então, enviada ao AWS Elastic Container Registry, que
mantém um histótico destas imagens, e a partir deste registry, uma
tarefa é iniciada e colocada num container, contruídos de modo a
complementar a imagem (expondo as portas esperadas pela imagem e pela
aplicação, por exemplo). Este container é gerado e gerenciado de forma
automática, tendo politicas de segurança associados a ele.

A partir deste ponto, o container se comunica com duas outras
ferramentas: um cluster serverless de banco de dados aurora,
compatível com MySQL 5, no AWS Relational Database Service, serviço
que gerencia as instâncias de banco de dados de forma automática, e
permite acesso a elas a partir de grupos de segurança
específicos. Nestes grupos de segurança estão os containers da
aplicação backend, o que permite a comunicação entre ambas as
aplicações.

No outro lado da aplicação backend, o AWS Elastic Load Balancer
oferece um ponto de comunicação entre o a rede privada, onde o backend
e o banco de dados se encontram, e a internet externa. Este serviço
redireciona as requisições HTTPS externas entre todas as instâncias da
aplicação que estejam rodando.

Alguns pontos a ser melhor desenvolvidos na infraestrutura são:
\begin{itemize}
\item Manter a instâncias da aplicação backend ativas e estáveis por
  um período maior, de modo que pelo menos uma possa responder a
  qualquer momento;
\item Usar o protocolo HTTPS, como está estabelecido nos requisitos de
  segurança;
\item Implementar outras pipelines, para escalar a aplicação e
  realizar o rollback a uma versão anterior;
\item Isolar os ambientes de teste e de produção;
\item Usar o serviço de gerenciamento de segredos para armazenar
  senhas de banco de dados, e;
\item Proteger a branch \textt{master} do repositório do Github, para
  que esta aceite novos commits apenas a partir de Pull Requests, uma
  vez que um commit novo nesta é o trigger para a execução da pipeline
  no Github Actions.
\end{itemize}

\chapter{Tecnologias}

Nesse capítulo, serão abordados as principais tecnologias e seus casos de uso no sistema.

\section{Front-end}
Para o desenvolvimento do \gls{frontend} da aplicação estamos utilizando a linguagem Javascript, a \gls{framework} React-Native e estilizando a interface com a biblioteca Native-Base.
Além destas tecnologias, utilizamos as seguintes ferramentas da \gls{framework} Expo:

\begin{itemize}
	\item \underline{Expo CLI}:
		\begin{itemize}
			\item Possibilita a criação de novos projetos de React-Native com estrutura básica inclusa;
			\item Possibilita a execução da aplicação em modo de desenvolvimento com recarregamento rápido, viabilizando a emulação no navegador, em dispositivos móveis (a partir do Expo Go) e em emuladores de dispositivos móveis;
			
			\item Possibilita fazer o \gls{build} dos arquivos binários \gls{apk} e \gls{ipa};
		\end{itemize}
		
			\item \underline{Expo Go}:
		\begin{itemize}
			\item Possibilita a execução da aplicação em desenvolvimento em dispositivos móveis antes do \gls{deploy}
		\end{itemize}
\end{itemize}

\section{Back-end}
Para o desenvolvimento do \gls{backend} da aplicação, foi utilizado a linguagem Java com o \gls{framework} Spring e todo seu ecossistema de tecnologia disponível.

As principais tecnologias do\gls{backend} são:

\begin{itemize}

	\item \underline{Lombok}: Automatiza métodos amplamente utilizados no java, tais como \textit{getters}, \textit{setters} e construtores.

	\item \underline{Hibernate}: Permite a conexão com o banco de dados através de uma abstração de alto nível.	
	
	\item \underline{Hibernate-Spatial}: É um complemento do hibernate, permitindo a integração com os dados do tipo espacial, permitindo geolocalização.
	
	\item \underline{Spring-Data}: É uma ferramenta do ecossistema Spring que permite uma forma mais prática de comunicar a \gls{api} com o banco de dados através do hibernate, fornecendo uma abstração mais prática do que a do Hibernate clássico.
	
	\item \underline{Flyway}: É um versionador de scripts de banco de dados.
	
	\item \underline{OAuth2}: É uma tecnologia que fornece um servidor de autenticação e autorização, garantindo alta segurança e alta escalabilidade.
	
	\item \underline{Spring Security}: É uma ferramenta do ecossistema Spring que fornece recursos necessários para proteger uma \gls{api}.
	
	\item \underline{Javamail}: É um plugin que permite o envio de email dentro da nossa própria \gls{api}.
	
	\item \underline{Swagger}: É um plugin que permite a documentação da \gls{api}.
	
\end{itemize}





% ----------------------------------------------------------
% Finaliza a parte no bookmark do PDF
% para que se inicie o bookmark na raiz
% e adiciona espaço de parte no Sumário
% ----------------------------------------------------------
\phantompart

% ----------------------------------------------------------
% ELEMENTOS PÓS-TEXTUAIS
% ----------------------------------------------------------
\postextual
% ----------------------------------------------------------

% ----------------------------------------------------------
% Referências bibliográficas
% ----------------------------------------------------------
% quando não esta utilizando biblatex tem que carregar as referencias aqui
\IfPackageLoaded{biblatex}{}{%
\bibliography{referencias,exemplos/abntex2-doc-abnt-6023}
}

\input{pos-glossario.tex}

\end{document}